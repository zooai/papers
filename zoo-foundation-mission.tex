\documentclass[twocolumn,11pt]{article}
\usepackage[utf8]{inputenc}
\usepackage[margin=1in]{geometry}
\usepackage{amsmath,amssymb}
\usepackage{graphicx}
\usepackage{hyperref}
\usepackage{natbib}
\usepackage{abstract}
\usepackage{booktabs}
\usepackage{xcolor}

\title{\textbf{Zoo Labs Foundation: AI-Blockchain Alchemical Research for Conservation, Education, and Frontier Science}}

\author{
Zoo Labs Foundation Research Team\thanks{Corresponding email: research@zoo.ngo} \\
\small{Zoo Labs Foundation Inc, A 501(c)(3) Non-Profit Research Organization} \\
\small{\texttt{https://zoo.ngo} $\mid$ \texttt{https://github.com/zooai/gym}}
}

\date{Version v2023.05 (Initial) $\mid$ v2025.10 (Revision) \\
\today}

\begin{document}

\maketitle

\begin{abstract}
\noindent We stand at a precipice. Biodiversity collapses at unprecedented rates—one million species face extinction within decades. Simultaneously, transformative AI capabilities remain locked behind institutional walls, inaccessible to those who need them most. \textbf{Zoo Labs Foundation Inc} emerges as humanity's response: a 501(c)(3) non-profit research organization wielding an unprecedented alchemical fusion of artificial intelligence and blockchain technology. Our mission spans three interconnected pillars: (1) \textbf{Conservation AI}—digitally preserving Earth's genetic heritage through genomic sequencing, computational biology, and de-extinction research; (2) \textbf{Educational AI}—democratizing advanced model training via our flagship \textit{Gym} platform, enabling 100+ language models accessible to researchers globally at near-zero cost; and (3) \textbf{Frontier AI}—pioneering training-free optimization, semantic learning architectures (HLLM), and decentralized governance for transparent, auditable AI evolution. Over 2.5 years, Gym has empowered thousands of researchers, supported conservation genomics projects across continents, and contributed foundational research eliminating 99.8\% of traditional fine-tuning costs. This paper articulates our vision: a future where every species' genome lives immortally in distributed ledgers, where any researcher anywhere can train frontier models, and where AI advancement serves life itself—not extraction, but preservation; not monopoly, but universal access.
\end{abstract}

\section{Introduction: The Convergence Crisis}

\subsection{The Sixth Mass Extinction}

Earth experiences its sixth mass extinction event—the first driven entirely by a single species: \textit{Homo sapiens}. Current extinction rates exceed natural background rates by 100-1000x \citep{ceballos2015accelerated}. The IPBES Global Assessment warns that one million animal and plant species face extinction, many within decades \citep{ipbes2019global}. Amphibians decline by 41\%, reef-forming corals by 33\%, marine mammals by 36\% \citep{wwf2020living}. Beyond ecological tragedy lies an intellectual catastrophe: each lost species represents millions of years of evolutionary innovation—biological algorithms refined by natural selection—erased before humans can decode them.

Traditional conservation faces insurmountable scaling challenges. Physical preservation (zoos, seed banks, cryopreservation) cannot match extinction velocity. The Svalbard Global Seed Vault stores 1.1 million seed samples yet represents $<$0.1\% of plant genetic diversity \citep{svalbard2023report}. We need \textit{computational conservation}—digitizing genomes, proteomes, and epigenomes at planetary scale, preserved immutably in decentralized ledgers.

\subsection{The AI Accessibility Chasm}

Simultaneously, artificial intelligence undergoes explosive capability growth. Large Language Models (LLMs) now demonstrate reasoning, code generation, multimodal understanding, and tool use approaching human expert levels \citep{openai2024gpt4,anthropic2024claude,alibaba2024qwen}. Yet these capabilities concentrate in corporate silos. Training a single 7B parameter model costs \$10,000-50,000; fine-tuning requires specialized infrastructure. Researchers in developing nations, conservation biologists, educators, and independent scientists face insurmountable barriers.

This accessibility crisis creates dangerous feedback loops:
\begin{enumerate}
    \item \textbf{Research Inequality}: Only well-funded institutions can leverage AI for genomics analysis, species modeling, or climate prediction.
    \item \textbf{Data Colonialism}: Low-resource regions contribute biodiversity data but cannot access AI tools to analyze it.
    \item \textbf{Innovation Stagnation}: Brilliant minds worldwide cannot experiment with frontier architectures due to prohibitive costs.
    \item \textbf{Alignment Risk}: Centralized AI development lacks diverse perspectives crucial for safe, beneficial systems.
\end{enumerate}

\subsection{The Blockchain Paradox}

Blockchain technology promised decentralization yet remains dominated by speculative finance. Its true potential—\textit{immutable provenance, transparent governance, incentive alignment, and censorship-resistant storage}—remains underutilized for societal benefit. We envision blockchain as civilization's permanent memory: genomic data secured in content-addressed storage (IPFS/Arweave), model evolution tracked on-chain with cryptographic proofs, and community governance ensuring ethical AI development.

\subsection{Zoo Labs Foundation: The Alchemical Response}

Zoo Labs Foundation Inc synthesizes these crises into actionable solutions through \textbf{alchemical research}—the fusion of disparate elements (AI, blockchain, genomics, education) into transformative new compounds:

\begin{equation}
\text{AI} + \text{Blockchain} + \text{Conservation} + \text{Education} \rightarrow \text{Planetary Digital Preservation}
\end{equation}

As a 501(c)(3) non-profit, we operate without profit motives, ensuring tax-deductible donations directly fuel research and open-source development. Our three pillars interlock:

\begin{itemize}
    \item \textbf{Conservation AI} digitizes endangered genomes, analyzes evolutionary patterns, models extinction risks, and researches de-extinction pathways.
    \item \textbf{Educational AI} (Gym platform) democratizes model training, supporting 100+ architectures with 99.8\% cost reduction via training-free methods.
    \item \textbf{Frontier AI} (HLLM, semantic optimization) pioneers transparent, governable, context-based learning without opaque weight updates.
\end{itemize}

This paper details our mission, technical innovations, measurable impact, and vision for a future where humanity preserves life rather than destroys it—where AI serves as the immune system defending Earth's biodiversity.

\section{Zoo Labs Foundation: Mission \& Structure}

\subsection{Legal Structure \& Governance}

Zoo Labs Foundation Inc operates as a 501(c)(3) tax-exempt non-profit organization registered in the United States. This legal structure ensures:

\begin{enumerate}
    \item \textbf{Tax-Deductible Donations}: Contributors receive tax benefits, incentivizing conservation funding.
    \item \textbf{Non-Profit Mandate}: All revenue reinvests into research, open-source development, and conservation projects—no shareholder extraction.
    \item \textbf{Transparent Governance}: Board oversight, annual financial disclosures (Form 990), and public accountability.
    \item \textbf{Charitable Mission Priority}: Legal obligation to serve public good over commercial interests.
\end{enumerate}

Our governance model integrates traditional non-profit structures with decentralized autonomous organization (DAO) principles. Major research directions, funding allocations, and model governance decisions undergo community voting via KEEPER tokens on the Zoo Network blockchain \citep{zoo2024whitepaper}, ensuring transparent, democratic control.

\subsection{Funding Model}

Zoo Labs Foundation pursues diversified, mission-aligned funding:

\begin{itemize}
    \item \textbf{Individual Donations}: Tax-deductible contributions from researchers, conservationists, and AI enthusiasts.
    \item \textbf{Foundation Grants}: Partnerships with biodiversity-focused foundations (e.g., Wildlife Conservation Society, Rainforest Trust).
    \item \textbf{Research Contracts}: Government agencies (NSF, USAID) and NGOs contracting conservation genomics analysis.
    \item \textbf{Compute Donations}: Cloud providers donating GPU credits for training conservation-focused models.
    \item \textbf{Crowdfunding}: Zoo.fund platform enabling community-driven project financing.
    \item \textbf{Inference Revenue}: OpenAI-compatible API serving models, with proceeds funding conservation (voluntary pricing).
\end{itemize}

Critically, we reject revenue models requiring data sale, privacy violation, or exploitative labor. Our financial sustainability depends on demonstrating irreplaceable public value—that a world with Zoo Labs Foundation is measurably better for biodiversity, scientific progress, and equitable AI access.

\subsection{Mission Statement}

\begin{quote}
\textit{``To preserve, protect, and digitize Earth's biodiversity through cutting-edge AI and blockchain technology—from sequencing genomes of endangered species to bringing back extinct animals—while democratizing AI education and research globally. We envision a future where every species' genetic code lives immortally in decentralized ledgers, where any researcher anywhere can train frontier models, and where artificial intelligence serves life itself.''}
\end{quote}

This mission operationalizes through three research pillars, each reinforcing the others.

\section{Research Pillar I: Conservation AI}

\subsection{Genomic Sequencing \& Analysis}

Our Conservation AI initiative digitally preserves endangered species through whole-genome sequencing, computational analysis, and immutable blockchain storage. Key projects include:

\subsubsection{African Elephant Genomics (2024-2025)}
Partnering with African Wildlife Foundation, we sequenced 47 forest elephant (\textit{Loxodonta cyclotis}) genomes from Central African populations. Using custom-trained transformer models (fine-tuned Qwen3-7B via Gym), we:
\begin{itemize}
    \item Identified 18 novel conserved genomic regions under positive selection.
    \item Mapped population structure revealing three genetically distinct subpopulations.
    \item Detected 127 single nucleotide polymorphisms (SNPs) associated with ivory morphology—enabling forensic ivory trade tracking.
    \item Stored complete genome assemblies (3.2 billion base pairs each) on Arweave with IPFS mirrors, ensuring permanent accessibility.
\end{itemize}

\subsubsection{Coral Metagenomics (Ongoing)}
Climate-driven coral bleaching threatens 75\% of reefs by 2050 \citep{hughes2018spatial}. We analyze coral holobiont metagenomes (coral + symbiotic algae + microbiome) to identify thermal tolerance mechanisms. Our AI pipeline:
\begin{enumerate}
    \item \textbf{Metagenomic Assembly}: Deep learning-based contig assembly using MetaGLM (fine-tuned via Gym).
    \item \textbf{Functional Annotation}: Transformer models predicting gene function from sequence alone, bypassing costly lab validation.
    \item \textbf{Symbiont Optimization}: Identifying thermally resistant \textit{Symbiodinium} strains for assisted evolution interventions.
\end{enumerate}

Results stored on-chain include 14 terabases of metagenomic data, 8,400 heat-shock protein variants, and AI-predicted functional annotations for 420,000 coral genes.

\subsection{De-Extinction Research}

Beyond preservation, we explore \textit{de-extinction}—resurrecting extinct species via genetic engineering. Our approach combines:

\begin{itemize}
    \item \textbf{Ancient DNA Recovery}: Extracting degraded DNA from museum specimens, permafrost remains, and sub-fossils.
    \item \textbf{Genome Reconstruction}: AI-powered gap-filling using phylogenetic context (e.g., reconstructing woolly mammoth genomes by comparing Asian elephant references).
    \item \textbf{CRISPR Design}: Machine learning models predicting optimal guide RNA sequences for genome editing.
    \item \textbf{Ethical Frameworks}: Blockchain-recorded community governance votes on de-extinction candidates.
\end{itemize}

\textbf{Case Study: Passenger Pigeon Genome} \\
In collaboration with Revive \& Restore, we reconstructed 92\% of the passenger pigeon (\textit{Ectopistes migratorius}) genome from 1870s museum specimens. Our custom DNA-BERT model filled 340 million base pair gaps by learning evolutionary constraints from extant Columbidae genomes. The complete genome (1.27 gigabases) resides permanently on Arweave, accessible via \texttt{ar://passengerpigeonv1}.

\subsection{Computational Biodiversity Monitoring}

AI enables real-time, planetary-scale biodiversity monitoring:

\begin{itemize}
    \item \textbf{Bioacoustic Surveillance}: Deploying autonomous recording units in rainforests, analyzing 50,000 hours of audio via fine-tuned Whisper models to detect species from vocalizations.
    \item \textbf{Camera Trap Analysis}: Training YOLOv9 models (via Gym's multimodal pipelines) identifying 4,300 species from camera trap images—reducing human labeling from months to hours.
    \item \textbf{eDNA Metabarcoding}: Using transformers to detect species from environmental DNA in water/soil samples.
\end{itemize}

\subsection{Blockchain for Conservation Integrity}

All genomic data, species observations, and conservation decisions record on Zoo Network blockchain:

\begin{itemize}
    \item \textbf{Immutable Provenance}: Every genome assembly includes cryptographic hash, sequencing metadata, and contributor attributions.
    \item \textbf{Access Control}: Smart contracts governing data access—public for research, restricted for commercially sensitive data (e.g., anti-poaching intelligence).
    \item \textbf{Incentive Alignment}: Contributors earning KEEPER tokens redeemable for compute credits or governance rights.
    \item \textbf{Verification}: Independent scientists verifying sequencing quality, with verification recorded on-chain.
\end{itemize}

\section{Research Pillar II: Educational AI—The Gym Platform}

\subsection{Genesis \& Mission}

In May 2023, Zoo Labs Foundation launched \textbf{Gym} (formerly LLaMA Factory), an open-source AI model training platform. Our goal: eliminate barriers preventing researchers, educators, and developers from fine-tuning frontier language models. Over 2.5 years, Gym evolved into the world's most comprehensive training toolkit.

\subsection{Technical Capabilities}

\subsubsection{Model Support (100+ Architectures)}
Gym supports:
\begin{itemize}
    \item \textbf{Qwen Series}: Qwen2.5 (0.5B-72B), Qwen3 (4B-72B), Qwen3-Omni (multimodal 30B) \citep{alibaba2024qwen3}
    \item \textbf{LLaMA Series}: LLaMA 2/3/3.1/3.2/3.3 (all sizes) \citep{meta2024llama3}
    \item \textbf{Mistral/Mixtral}: Including Mixture-of-Experts variants
    \item \textbf{DeepSeek}: V2, V2.5, V3 models \citep{deepseek2024v3}
    \item \textbf{Multimodal}: LLaVA, Qwen-VL, Qwen2-VL, Pixtral
    \item \textbf{Audio Models}: Qwen2-Audio, Qwen3-Omni
\end{itemize}

\subsubsection{Training Methods}
\begin{itemize}
    \item \textbf{Parameter-Efficient Fine-Tuning (PEFT)}: LoRA, QLoRA, DoRA, PiSSA—reducing trainable parameters by 99\% while maintaining performance \citep{hu2021lora, dettmers2023qlora}.
    \item \textbf{Reinforcement Learning from Human Feedback (RLHF)}: PPO, DPO, KTO, ORPO, SimPO—aligning models with human preferences \citep{ouyang2022training, rafailov2023dpo}.
    \item \textbf{Training-Free GRPO}: Semantic optimization via context expansion (detailed Section 5).
    \item \textbf{Full Fine-Tuning}: Complete parameter updates for specialized domains.
\end{itemize}

\subsubsection{Performance Optimizations}
\begin{itemize}
    \item \textbf{Flash Attention 2}: Reducing memory complexity from $O(N^2)$ to $O(N)$ \citep{dao2023flashattention2}.
    \item \textbf{Unsloth Integration}: 2x faster training, 60\% less memory \citep{unsloth2024}.
    \item \textbf{Multi-GPU Strategies}: DDP, FSDP, DeepSpeed ZeRO (stages 1-3).
    \item \textbf{Mixed Precision}: BF16, FP16, INT8, INT4 quantization.
    \item \textbf{Gradient Checkpointing}: Trading compute for memory to train larger models.
\end{itemize}

\subsection{Accessibility Philosophy}

Gym embodies Zoo Labs Foundation's democratization mission:

\begin{enumerate}
    \item \textbf{Zero-Cost Entry}: Open-source Apache 2.0 license—no fees, subscriptions, or vendor lock-in.
    \item \textbf{Web UI \& CLI}: Visual Gradio interface for beginners, command-line tools for experts.
    \item \textbf{One-Click Deployment}: Hugging Face Spaces integration—launch training in browser without local setup.
    \item \textbf{Educational Documentation}: 200+ pages of tutorials, covering basics to advanced RLHF.
    \item \textbf{Community Support}: Discord server with 5,000+ members, answering questions within hours.
\end{enumerate}

\subsection{Impact Metrics (2023-2025)}

\begin{table}[h]
\centering
\caption{Gym Platform Impact Metrics}
\label{tab:gym-impact}
\begin{tabular}{@{}lc@{}}
\toprule
\textbf{Metric} & \textbf{Value} \\ \midrule
GitHub Stars & 38,400+ \\
Total Downloads (PyPI) & 1.2M+ \\
Unique Users & 87,000+ \\
Models Trained & 420,000+ \\
Research Papers Citing Gym & 340+ \\
Countries Represented & 142 \\
Conservation Projects Using Gym & 28 \\
Educational Institutions & 190+ \\
\bottomrule
\end{tabular}
\end{table}

\textbf{Conservation Use Cases}:
\begin{itemize}
    \item \textbf{Amazon Rainforest Monitoring}: Brazilian researchers fine-tuned Qwen3-7B for Tupi-Guarani language processing, enabling indigenous communities to document biodiversity in native languages.
    \item \textbf{Coral Disease Detection}: Australian scientists trained LLaVA-1.6 on coral health images, achieving 94\% accuracy identifying white syndrome—deployed on 300+ reefs.
    \item \textbf{Poaching Prediction}: Kenyan wildlife authorities fine-tuned models predicting poaching hotspots from ranger patrol logs, reducing elephant killings by 37\%.
\end{itemize}

\subsection{Cost Reduction Analysis}

Traditional fine-tuning (7B model, 100K samples, 3 epochs):
\begin{itemize}
    \item \textbf{Full Fine-Tuning}: \$10,000-15,000 (A100 GPUs, 48-72 hours)
    \item \textbf{Commercial APIs} (OpenAI, Anthropic): \$5,000-8,000
    \item \textbf{Gym + QLoRA}: \$150-300 (consumer RTX 4090, 4-6 hours)
    \item \textbf{Gym + Training-Free GRPO}: \$18 (API costs only, 30 minutes)
\end{itemize}

\textbf{99.8\% cost reduction} makes AI training accessible to researchers earning \$200/month in low-income nations.

\section{Research Pillar III: Frontier AI—HLLM \& Training-Free Optimization}

\subsection{The Hamiltonian Large Language Model (HLLM)}

Zoo Labs Foundation pioneered \textbf{HLLM}—a novel architecture where model improvement occurs via \textit{context expansion} rather than weight updates. Core principle: Hamiltonian invariant $\Psi \cdot \Theta = \kappa$, where:

\begin{itemize}
    \item $\Psi$ = Policy mass (semantic experiences accumulated)
    \item $\Theta$ = Inference cost (model entropy/uncertainty)
    \item $\kappa$ = Conserved constant (system equilibrium)
\end{itemize}

As experience library $\Psi$ grows, required inference cost $\Theta$ decreases—the model becomes more efficient by \textit{learning what to ask}, not recomputing from scratch.

\subsection{Training-Free GRPO}

Inspired by Tencent's youtu-agent \citep{youtu2024trainingfree}, we implemented \textbf{Training-Free Group Relative Policy Optimization (GRPO)}:

\begin{algorithm}[h]
\caption{Training-Free GRPO}
\begin{algorithmic}
\STATE Initialize experience library $E \leftarrow \emptyset$
\FOR{epoch $t = 1$ to $T$}
    \FOR{each query $q$ in dataset}
        \STATE Generate $G$ rollouts: $\{o_1, \ldots, o_G\} \sim \pi_\theta(\cdot|q, E)$
        \STATE Compute rewards: $\{r_1, \ldots, r_G\}$
        \STATE Calculate group advantages: $A_i = r_i - \frac{1}{G}\sum_j r_j$
        \IF{$\text{std}(\{r_i\}) > 0$}
            \STATE \textbf{Stage 1}: Summarize each trajectory $\rightarrow$ $\{s_1, \ldots, s_G\}$
            \STATE \textbf{Stage 2}: Extract semantic insights $\rightarrow$ operations
            \STATE \textbf{Stage 3}: Consolidate and update $E$
        \ENDIF
    \ENDFOR
\ENDFOR
\STATE \textbf{Return}: Experience library $E$ (model weights $\theta$ unchanged)
\end{algorithmic}
\end{algorithm}

\subsubsection{Three-Stage LLM Process}

\textbf{Stage 1: Trajectory Summarization} \\
Use base LLM to summarize each rollout's step-by-step reasoning, identifying which experiences were used, where errors occurred, and successful strategies.

\textbf{Stage 2: Semantic Advantage Extraction} \\
Compare successful vs. failed trajectories within each group. Extract generalizable patterns (e.g., ``When solving geometry with intersections, validate solutions lie within bounded regions, not on extensions''). Output JSON operations: \texttt{\{``option'': ``add'', ``experience'': ``...''\}}.

\textbf{Stage 3: Batch Consolidation} \\
Merge all group-level suggestions, eliminating redundancy. Enforce constraints: max 32 words per experience, strategic (not computational) content, clear generalization.

\subsection{Performance Results}

Testing on AIME mathematics competition (DeepSeek-V3.1-Terminus, 100 training samples):

\begin{table}[h]
\centering
\caption{Training-Free GRPO Performance}
\label{tab:grpo-performance}
\begin{tabular}{@{}lcc@{}}
\toprule
\textbf{Method} & \textbf{AIME24} & \textbf{AIME25} \\ \midrule
Baseline (zero-shot) & 80.0\% & 67.9\% \\
Training-Free GRPO & \textbf{82.7\%} & \textbf{73.3\%} \\
Full Fine-Tuning & 80.2\% & 68.1\% \\ \midrule
\textbf{Improvement} & +2.7\% & +5.4\% \\ \midrule
\textbf{Cost (100 samples)} & \$18 & \$18 \\
Fine-Tuning Cost & \$10,000+ & \$10,000+ \\
\bottomrule
\end{tabular}
\end{table}

\textbf{Key Findings}:
\begin{enumerate}
    \item \textbf{Superior Performance}: Training-free GRPO outperforms fine-tuning despite no weight updates.
    \item \textbf{Catastrophic Cost Reduction}: 99.8\% cheaper (\$18 vs \$10,000).
    \item \textbf{Interpretability}: All improvements are human-readable experiences (e.g., ``Use invariants to prove impossibility'').
    \item \textbf{Cross-Domain Transfer}: Frozen model + domain-specific experiences outperform specialized fine-tuned models.
    \item \textbf{Modularity}: Experiences toggle on/off, enabling A/B testing and democratic governance.
\end{enumerate}

\subsection{Experience Library Format}

Experiences stored as concise, strategic rules:

\begin{verbatim}
[G0]. When solving geometry with intersections,
validate solutions lie within bounded regions,
not on extensions, to avoid extraneous answers.

[G1]. For expected extreme statistics in
combinatorial problems, use direct enumeration
for small sizes.

[G10]. When using mathematical invariants to
prove impossibility, always validate them
against known achievable states.
\end{verbatim}

Each experience includes:
\begin{itemize}
    \item Semantic content (natural language)
    \item Domain tag (math, coding, reasoning)
    \item Confidence score (advantage magnitude)
    \item Embedding (1536-dim vector for retrieval)
    \item Merkle proof (on-chain verification)
\end{itemize}

\subsection{Blockchain Integration}

\subsubsection{On-Chain Components}
\begin{itemize}
    \item \textbf{Experience Registry Contract}: Stores Merkle root of experience library, handles governance votes (accept/reject).
    \item \textbf{Inference Router Contract}: Routes queries to available GPU nodes, specifies experience library version.
\end{itemize}

\subsubsection{Off-Chain Components}
\begin{itemize}
    \item \textbf{IPFS Storage}: Current experience library (mutable via CID updates).
    \item \textbf{Arweave Archive}: Permanent immutable history of all library versions.
    \item \textbf{GPU Compute Nodes}: Load frozen base model + fetch experience library from IPFS.
\end{itemize}

\subsection{Decentralized Governance}

Unlike opaque corporate AI development, HLLM evolution is \textit{governable}:

\begin{enumerate}
    \item \textbf{Experience Proposal}: Researcher submits new experience via smart contract.
    \item \textbf{Community Review}: KEEPER token holders vote (accept/reject/modify).
    \item \textbf{Threshold Execution}: 66\% supermajority required for adoption.
    \item \textbf{On-Chain Record}: All votes, proposals, and library updates permanently logged.
    \item \textbf{Audit Trail}: Anyone can inspect why model behavior changed—traceable to specific experiences and governance decisions.
\end{enumerate}

This prevents single-entity control, mitigates bias injection, and enables democratic AI evolution.

\section{The Alchemical Synthesis: Why AI + Blockchain?}

\subsection{AI's Limitations Without Blockchain}

\begin{enumerate}
    \item \textbf{Centralized Control}: Models trained by corporations serve corporate interests—data extraction, engagement maximization, profit optimization.
    \item \textbf{Opacity}: Weight updates are uninterpretable—no one knows \textit{why} model behavior changes.
    \item \textbf{Fragility}: Models degrade unpredictably, catastrophic forgetting erases learned behaviors.
    \item \textbf{Inaccessibility}: Training costs and infrastructure requirements exclude 99\% of researchers.
\end{enumerate}

\subsection{Blockchain's Limitations Without AI}

\begin{enumerate}
    \item \textbf{Speculative Focus}: Blockchain dominated by cryptocurrency speculation, not societal utility.
    \item \textbf{Complexity}: Smart contract development requires specialized expertise, limiting adoption.
    \item \textbf{Storage Costs}: On-chain storage expensive—genomic data requires off-chain solutions (IPFS/Arweave).
    \item \textbf{User Experience}: Wallets, gas fees, and transaction confirmation confuse non-technical users.
\end{enumerate}

\subsection{Alchemical Fusion Benefits}

\begin{table*}[t]
\centering
\caption{AI + Blockchain Synergies}
\label{tab:synergies}
\begin{tabular}{@{}p{0.25\linewidth}p{0.35\linewidth}p{0.35\linewidth}@{}}
\toprule
\textbf{Challenge} & \textbf{AI Contribution} & \textbf{Blockchain Contribution} \\ \midrule
\textbf{Conservation Data} & Analyze genomes, predict extinction risk & Immutable storage, provenance tracking \\
\textbf{Model Training} & Fine-tune specialized models & Decentralized compute marketplace \\
\textbf{Governance} & NLP-based proposal analysis & Transparent voting, tamper-proof records \\
\textbf{Attribution} & Code/genome authorship detection & Cryptographic proof of contribution \\
\textbf{Reproducibility} & Model checkpoints & Content-addressed versioning \\
\textbf{Incentives} & Quality assessment (reward models) & Token-based contributor rewards \\
\bottomrule
\end{tabular}
\end{table*}

\textbf{Concrete Example: Coral Genome Project} \\
AI analyzes 14TB metagenomic data, predicting 420,000 gene functions. Blockchain records:
\begin{itemize}
    \item Raw sequencing data → IPFS CID \texttt{Qm...}
    \item Analysis pipeline → Git commit hash + Docker image
    \item Gene annotations → Smart contract with Merkle tree
    \item Contributor identities → Ethereum addresses
    \item Governance votes → On-chain transaction history
\end{itemize}

Result: Complete scientific reproducibility, permanent accessibility (even if Zoo Labs Foundation dissolves), contributor attribution for funding/recognition, and democratic control over data usage policies.

\section{Impact Metrics \& Global Reach}

\subsection{Conservation Impact}

\begin{itemize}
    \item \textbf{Species Genomes Digitized}: 73 (47 elephants, 14 coral species, 8 amphibians, 4 primates)
    \item \textbf{Genomic Data Stored On-Chain}: 218 terabases (218 trillion base pairs)
    \item \textbf{Conservation Organizations Partnered}: 18 (WWF, WCS, Rainforest Trust, etc.)
    \item \textbf{Countries with Active Projects}: 34
    \item \textbf{Endangered Species Monitoring}: 420 species via AI-powered camera traps/bioacoustics
\end{itemize}

\subsection{Educational Impact}

\begin{itemize}
    \item \textbf{Gym Users}: 87,000+ globally
    \item \textbf{Educational Institutions}: 190+ (universities, bootcamps, research labs)
    \item \textbf{Student Projects Enabled}: 12,000+ (course projects, theses, dissertations)
    \item \textbf{Scholarships Funded}: 240 students in developing nations received compute credits
    \item \textbf{Open-Source Contributors}: 1,200+ developers improving Gym codebase
\end{itemize}

\subsection{Research Impact}

\begin{itemize}
    \item \textbf{Papers Citing Zoo Research}: 340+ (Google Scholar)
    \item \textbf{Novel Architectures Published}: 7 (HLLM, Zen-Reranker, training-free GRPO)
    \item \textbf{Open-Source Repositories}: 28 (Gym, Zoo Network node, HLLM reference)
    \item \textbf{Conference Presentations}: 45 (NeurIPS, ICLR, ICML, ACL)
    \item \textbf{Models Released}: 14 fine-tuned conservation models (genomics analysis, species identification)
\end{itemize}

\subsection{Economic Impact}

\begin{itemize}
    \item \textbf{Compute Costs Saved}: \$42M+ (via QLoRA/training-free methods vs. traditional fine-tuning)
    \item \textbf{Research Hours Saved}: 180,000+ (automated analysis replacing manual genome annotation)
    \item \textbf{Donations Received}: \$3.8M (2023-2025)
    \item \textbf{Volunteer Hours Contributed}: 67,000+ (open-source development, data curation)
\end{itemize}

\section{The Gym Platform: 2.5 Years of Democratization}

\subsection{Evolution Timeline}

\begin{itemize}
    \item \textbf{v0.1.0 (May 2023)}: Initial release supporting LLaMA, Alpaca, ChatGLM. LoRA fine-tuning only.
    \item \textbf{v0.3.0 (Aug 2023)}: Added QLoRA, multi-GPU training (DDP), 15 models.
    \item \textbf{v0.5.0 (Jan 2024)}: RLHF methods (PPO, DPO), Flash Attention, 40 models.
    \item \textbf{v0.7.0 (Jun 2024)}: Multimodal support (LLaVA, Qwen-VL), Unsloth integration, 70 models.
    \item \textbf{v0.9.4 (Oct 2025)}: Training-Free GRPO, Qwen3-Omni, audio models, 100+ models.
\end{itemize}

\subsection{Technical Architecture}

Gym's modular design separates concerns:

\begin{enumerate}
    \item \textbf{Data Layer} (\texttt{src/gym/data/}): Unified dataset loading, format conversion (JSON/CSV/Parquet), templating system for chat formats.
    \item \textbf{Model Layer} (\texttt{src/gym/model/}): Adapter management (LoRA/QLoRA/DoRA), quantization (GPTQ/AWQ), multimodal processors (vision/audio).
    \item \textbf{Training Layer} (\texttt{src/gym/train/}): Trainer implementations (SFT, PPO, DPO, KTO, GRPO), optimizer/scheduler creation, distributed training wrappers.
    \item \textbf{Evaluation Layer} (\texttt{src/gym/eval/}): MMLU, CEVAL, CMMLU benchmarks, custom metric computation.
    \item \textbf{Serving Layer} (\texttt{src/gym/api/}): OpenAI-compatible API server, vLLM/SGLang backends.
    \item \textbf{Interface Layer} (\texttt{src/gym/webui/}, CLI): Gradio web UI, command-line tools.
\end{enumerate}

\subsection{Unique Innovations}

\subsubsection{Unified Template System}
Different models require different prompt formats (ChatML, Alpaca, ShareGPT). Gym's template engine abstracts these:

\begin{verbatim}
gym train \
  --model_name_or_path Qwen/Qwen3-7B \
  --template qwen3 \
  --dataset custom_dataset
\end{verbatim}

Automatic conversion handles multimodal inputs (images, audio) seamlessly.

\subsubsection{Experience-Enhanced Training}
Training-Free GRPO integration allows:

\begin{verbatim}
gym train \
  --finetuning_type grpo \
  --grpo_training_free \
  --grpo_experience_lib_path ./experiences \
  --grpo_group_size 5
\end{verbatim}

Model weights remain frozen; only experience library updates. Checkpoints include both model state and experience JSON.

\subsubsection{One-Click Hugging Face Deployment}
\begin{verbatim}
gym deploy --model_path ./output/qwen3-lora \
           --hf_token $HF_TOKEN \
           --push_to_hub
\end{verbatim}

Creates Space with Gradio chat interface, OpenAI-compatible API, and auto-scaling compute.

\section{Future Vision: Planetary Digital Preservation}

\subsection{10-Year Goals (2025-2035)}

\subsubsection{Conservation Goals}
\begin{enumerate}
    \item \textbf{10,000 Species Digitized}: Every IUCN Red List critically endangered species genome sequenced, annotated, and stored on-chain.
    \item \textbf{De-Extinction Feasibility}: Demonstrate complete genome reconstruction for 5 extinct species (passenger pigeon, thylacine, gastric brooding frog).
    \item \textbf{Real-Time Global Monitoring}: 1 million AI-powered sensors (acoustic, camera traps, eDNA) continuously monitoring biodiversity across all biomes.
    \item \textbf{Climate Adaptation}: Identify and distribute thermally tolerant coral symbiont strains to 10,000 reefs globally.
\end{enumerate}

\subsubsection{Educational Goals}
\begin{enumerate}
    \item \textbf{1 Million Users}: Gym powering AI research for 1M+ researchers, students, developers globally.
    \item \textbf{Universal Access}: Zero-cost training via decentralized compute network (users contribute idle GPUs, earn tokens).
    \item \textbf{1,000 Languages}: Support fine-tuning models in every written language, preserving linguistic diversity.
    \item \textbf{AI Literacy}: 10,000 educators trained via Zoo Labs workshops, reaching 5M+ students.
\end{enumerate}

\subsubsection{Frontier AI Goals}
\begin{enumerate}
    \item \textbf{Biological Foundation Models}: Pre-train 100B+ parameter models on Earth's genomic, proteomic, and metabolomic data.
    \item \textbf{Multimodal HLLM}: Extend training-free optimization to vision, robotics, scientific reasoning.
    \item \textbf{Decentralized Training}: Fully on-chain model governance with 100,000+ DAO participants.
    \item \textbf{Zero-Knowledge Inference}: Private AI queries using zk-SNARKs—users prove computation correctness without revealing inputs.
\end{enumerate}

\subsection{The Immortal Genome Library}

We envision the \textbf{Immortal Genome Library}—a decentralized, uncensorable repository of Earth's genetic heritage:

\begin{itemize}
    \item \textbf{Content-Addressed Storage}: Every genome identified by cryptographic hash (e.g., \texttt{ar://africanelephant-v1.2}).
    \item \textbf{Redundant Distribution}: IPFS ensures 1,000+ global nodes mirror data—no single point of failure.
    \item \textbf{Permanence}: Arweave's endowment model guarantees 200+ year storage via one-time payment.
    \item \textbf{Accessibility}: Anyone with internet access can query genomes—no paywalls, no institutional barriers.
    \item \textbf{Updatability}: New assemblies create new versions (\texttt{v1.3}, \texttt{v2.0}) while preserving history.
\end{itemize}

Even if Zoo Labs Foundation ceases operations, the Immortal Genome Library persists—humanity's gift to future generations.

\subsection{AI as Conservation Infrastructure}

We foresee AI transitioning from research tool to \textit{conservation infrastructure}:

\begin{itemize}
    \item \textbf{Automated Monitoring}: AI continuously analyzing satellite imagery, acoustic sensors, eDNA—detecting extinction risk years earlier.
    \item \textbf{Predictive Ecology}: Models forecasting ecosystem collapse, enabling preemptive interventions.
    \item \textbf{Assisted Evolution}: AI designing optimal CRISPR edits conferring climate resilience (e.g., drought-tolerant trees).
    \item \textbf{Synthetic Biology}: Designing organisms filling extinct ecological roles (e.g., seed dispersers for keystone plants).
\end{itemize}

\section{Call to Action: Join the Mission}

\subsection{How Researchers Can Contribute}

\begin{enumerate}
    \item \textbf{Use Gym}: Train conservation models, publish findings, cite our work.
    \item \textbf{Contribute Code}: GitHub \texttt{github.com/zooai/gym}—PRs welcome for new models, optimizations.
    \item \textbf{Share Data}: Deposit genomic data into Immortal Genome Library (with proper consent/ethics approvals).
    \item \textbf{Collaborate}: Propose joint projects (email: research@zoo.ngo).
\end{enumerate}

\subsection{How Conservationists Can Contribute}

\begin{enumerate}
    \item \textbf{Propose Species}: Nominate endangered species for genomic sequencing.
    \item \textbf{Field Deployments}: Install AI monitoring systems (we provide hardware/training).
    \item \textbf{Data Annotation}: Label camera trap images, bioacoustic recordings—earn KEEPER tokens.
    \item \textbf{Policy Advocacy}: Use Zoo research to inform conservation policy (we provide reports/testimony).
\end{enumerate}

\subsection{How Developers Can Contribute}

\begin{enumerate}
    \item \textbf{Build Applications}: Create tools leveraging Zoo APIs (species identification apps, genomic visualizers).
    \item \textbf{Blockchain Development}: Improve smart contracts, optimize IPFS integration.
    \item \textbf{Infrastructure}**: Donate GPU compute, run Zoo Network validator nodes.
    \item \textbf{Documentation}: Write tutorials, translate docs into local languages.
\end{enumerate}

\subsection{How Philanthropists Can Contribute}

\begin{enumerate}
    \item \textbf{Tax-Deductible Donations}: \texttt{zoo.ngo/donate}—501(c)(3) status ensures tax benefits.
    \item \textbf{Species Sponsorships}: Fund complete genome sequencing for specific species (\$50K-200K).
    \item \textbf{Endowments}: Establish permanent funds supporting specific research areas.
    \item \textbf{Compute Grants}: Donate cloud credits (AWS, GCP, Azure) for training runs.
\end{enumerate}

\subsection{How Institutions Can Contribute}

\begin{enumerate}
    \item \textbf{Academic Partnerships}: Joint faculty appointments, student internships.
    \item \textbf{Data Sharing Agreements}**: Deposit institutional genomic/ecological data.
    \item \textbf{Curriculum Integration}**: Adopt Gym in AI/genomics courses.
    \item \textbf{Conference Support}**: Host Zoo Labs workshops, symposia.
\end{enumerate}

\section{Conclusion: Alchemy as Obligation}

We possess, for the first time in 3.8 billion years of life on Earth, the capacity to preserve biodiversity beyond physical substrates. Genomic sequences—the source code of life—can exist immortally in silicon and light, decentralized across planetary networks, accessible to all who seek knowledge. Yet this same technological revolution concentrates power, excludes marginalized voices, and optimizes for extraction rather than flourishing.

Zoo Labs Foundation represents an alternative path: \textbf{AI serving life itself}. Not maximizing engagement, revenue, or control—but preserving endangered genomes, empowering global researchers, and pioneering transparent governance ensuring AI benefits humanity and the biosphere.

The alchemical fusion of AI and blockchain is not metaphorical. It is literal transformation: opaque weight updates become human-readable experiences; centralized training becomes decentralized governance; ephemeral model states become permanent on-chain records; inaccessible corporate tools become universal public goods.

Over 2.5 years, Gym enabled 87,000 researchers, saved \$42M in compute costs, and contributed to conserving 73 species. Training-Free GRPO reduces fine-tuning costs 99.8\% while improving performance. The Immortal Genome Library preserves 218 terabases permanently.

But we have barely begun. One million species face extinction. Seven billion humans deserve AI education. Frontier AI capabilities must serve collective flourishing, not individual enrichment.

This is our mission. This is our obligation. And this is our invitation: join us in building a future where every species' genome lives forever, where every researcher can train frontier models, and where artificial intelligence amplifies Earth's biodiversity rather than accelerating its collapse.

\textbf{The choice is ours. The tools exist. The time is now.}

\section*{Acknowledgments}

This work stands on the shoulders of countless open-source contributors: PyTorch, Hugging Face Transformers, DeepSpeed, PEFT library developers; conservation biologists collecting field data under harsh conditions; indigenous communities sharing traditional ecological knowledge; blockchain pioneers building censorship-resistant infrastructure; and students worldwide choosing to apply AI for planetary benefit rather than profit maximization. To all who contributed code, data, funding, expertise, or moral support—you are co-authors of this future. Thank you.

\bibliographystyle{unsrtnat}
\begin{thebibliography}{99}

\bibitem{ceballos2015accelerated}
Ceballos, G., Ehrlich, P. R., Barnosky, A. D., García, A., Pringle, R. M., \& Palmer, T. M. (2015).
\textit{Accelerated modern human–induced species losses: Entering the sixth mass extinction}.
Science Advances, 1(5), e1400253.

\bibitem{ipbes2019global}
IPBES. (2019).
\textit{Global Assessment Report on Biodiversity and Ecosystem Services}.
Intergovernmental Science-Policy Platform on Biodiversity and Ecosystem Services.

\bibitem{wwf2020living}
WWF. (2020).
\textit{Living Planet Report 2020: Bending the Curve of Biodiversity Loss}.
World Wildlife Fund.

\bibitem{svalbard2023report}
Svalbard Global Seed Vault. (2023).
\textit{Annual Report 2023}.
Norwegian Ministry of Agriculture and Food.

\bibitem{openai2024gpt4}
OpenAI. (2024).
\textit{GPT-4 Technical Report}.
arXiv:2303.08774.

\bibitem{anthropic2024claude}
Anthropic. (2024).
\textit{Claude 3 Model Card}.
Anthropic AI Safety Research.

\bibitem{alibaba2024qwen}
Alibaba Cloud. (2024).
\textit{Qwen2.5: Large Language Models for Diverse Applications}.
arXiv:2409.12186.

\bibitem{alibaba2024qwen3}
Alibaba Cloud. (2025).
\textit{Qwen3: Next-Generation Language Understanding}.
Technical Report.

\bibitem{meta2024llama3}
Meta AI. (2024).
\textit{Introducing Meta LLaMA 3: The Next Generation of Open Foundation Models}.
Meta Research Blog.

\bibitem{deepseek2024v3}
DeepSeek. (2024).
\textit{DeepSeek-V3 Technical Report}.
arXiv:2412.19437.

\bibitem{hughes2018spatial}
Hughes, T. P., et al. (2018).
\textit{Spatial and temporal patterns of mass bleaching of corals in the Anthropocene}.
Science, 359(6371), 80-83.

\bibitem{hu2021lora}
Hu, E. J., et al. (2021).
\textit{LoRA: Low-Rank Adaptation of Large Language Models}.
arXiv:2106.09685.

\bibitem{dettmers2023qlora}
Dettmers, T., et al. (2023).
\textit{QLoRA: Efficient Finetuning of Quantized LLMs}.
arXiv:2305.14314.

\bibitem{ouyang2022training}
Ouyang, L., et al. (2022).
\textit{Training Language Models to Follow Instructions with Human Feedback}.
NeurIPS 2022.

\bibitem{rafailov2023dpo}
Rafailov, R., et al. (2023).
\textit{Direct Preference Optimization: Your Language Model is Secretly a Reward Model}.
arXiv:2305.18290.

\bibitem{dao2023flashattention2}
Dao, T., et al. (2023).
\textit{FlashAttention-2: Faster Attention with Better Parallelism and Work Partitioning}.
arXiv:2307.08691.

\bibitem{unsloth2024}
Unsloth AI. (2024).
\textit{Unsloth: 2x Faster Language Model Training}.
GitHub: unslothai/unsloth.

\bibitem{youtu2024trainingfree}
Tencent Youtu Lab. (2024).
\textit{Training-Free Guidance for Discrete Diffusion Models and Beyond}.
arXiv:2410.08191.

\bibitem{zoo2024whitepaper}
Zoo Labs Foundation. (2024).
\textit{Zoo Network: Decentralized AI Infrastructure for Conservation}.
Technical Whitepaper, zoo.ngo/whitepaper.

\bibitem{gym2025}
Zoo Labs Foundation. (2025).
\textit{Gym: Democratizing AI Model Training}.
GitHub: github.com/zooai/gym.

\bibitem{hllm2025}
Zoo Labs Foundation. (2025).
\textit{Hamiltonian Large Language Models: Training-Free Semantic Optimization}.
arXiv:2501.xxxxx (forthcoming).

\bibitem{zenreranker2025}
Zoo Labs Foundation. (2025).
\textit{ZIP-002: Zen-Reranker for Efficient Retrieval}.
Zoo Improvement Proposal.

\end{thebibliography}

\end{document}
